
\section{Resisting capitalist control over resources}

\subsection{Directly elect the Wikimedia Foundation Board}

\subsection{Fight paid editing}

There are efforts to get paid promotional editing out of Wikipedia, and undeclared paid advertising is already prohibited.  Specifically, a rule on English Wikipedia reads: "If you are paid in any way for contributing to Wikipedia, you \textbf{must} disclose it."\footnote{\url{https://en.wikipedia.org/wiki/Wikipedia:Paid-contribution_disclosure}} Another policy reads, "Do not edit Wikipedia in your own interests or in the interests of your external relationships."\footnote{\url{https://en.wikipedia.org/wiki/Wikipedia:Conflict_of_interest}} Rules exist to be broken, so paid editing continues, both declared and undeclared.

\subsection{Fighting capitalist influence on knowledge}

[see fight for zero, above]

\subsection{Fighting freeloading capitalist use of knowledge}

The CC-SA and CC0 licenses allow virtually unlimited capitalist reuse of our knowledge.

We may have the alternative to use the non-commercial CC BY-NC-SA license, although [I don't know] this might not protect from indirect uses like training AIs owned by capitalists [cite showing that this might help, but also repeat lessons learned about for-profit unis, etc.]

With no requirement to give back or be accountable to the public, we don't know what the social impact of capitalist reuse might be.

\subsection{Expand the types and sources of knowledge}

If we believe in the wiki model, then more types of knowledge should benefit from it.  For example, oral histories could be captured digitally and made available as primary sources.\footnote{Peter Gallert has made arguments for the inclusion of oral knowledge, see \url{https://blog.wikimedia.org/2014/12/12/tapping-into-the-knowledge-of-indigenous-communities/}}

\subsection{Fighting for workers' rights: an Editor's Union}

In negotiations between the contributors and entities like the Wikimedia Foundation, it's clear that the editors are at a disadvantage for not having any coherent, collective voice or ability to bargain.  A typical solution would be to create an association of Wikimedians, which would have the legal right to act on behalf of its constituents.  This proposal first surfaced at the time the Wikimedia Foundation was created\footnote{\url{https://meta.wikimedia.org/wiki/Association_of_Wikimedians}}, and comes up on the mailing lists, usually during crises.

\subsection{Fight against censorship and for net neutrality}

State and capitalist censorship or shaping of information is already a problem.

Slight changes in Google's search engine results pages can cause our site traffic to drop dramatically.

\subsection{Mitigate scarcity among preconditions}

[work backlogs and their side-effects]

\subsection{Counteract biases in contributor demographics}

\subsection{Encourage coordination between workers}


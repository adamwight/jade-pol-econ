\documentclass[format=sigconf,11pt]{acmart}
\usepackage{geometry}
\geometry{a4paper}
%\geometry{landscape}                % Activate for for rotated page geometry
%\usepackage[parfill]{parskip}    % Activate to begin paragraphs with an empty line rather than an indent
%\usepackage{graphicx}
%\usepackage{amssymb}
\usepackage{amsfonts}
\usepackage{epstopdf}
\usepackage{hyperref}
\usepackage{cleveref}
%\DeclareGraphicsRule{.tif}{png}{.png}{`convert #1 `dirname #1`/`basename #1 .tif`.png}

\title{[TITLE]}
\author{Adam Wight}
\email{awight@wikimedia.org}
\affiliation{Wikimedia Foundation}
%\date{}                                           % Activate to display a given date or no date

\crefformat{footnote}{#2\footnotemark[#1]#3}

\begin{document}
\maketitle
\section{Introduction}

[THESIS]

[WP is both embedded within capitalism and affected by it, and works against it through free knowledge dissemination, nonvocational education, and consensus principles.  At the end of each section, some suggestions are made about how we might fight to expropriate the resource.

I've tried to organize the subsections roughly in parallel, following the flow of each commodity.

Author funding sources: I was paid to write this paper, no more than \$850 of normal wages from the Wikimedia Foundation and nothing else.

-A. Wight, Bogot�, DC, Columbia

\section{Frequently asked questions}

\subsection{Money}

This story begins with money only to say that it's not a dominant force in the Wikimedia movement.  The majority of the Wikimedia Foundation's revenue, for example, comes from individual donations under \$30.  Almost none [cn] of this revenue goes towards directly producing content.

\subsubsection{Where does the money come from?}

\begin{itemize}
\item Shadow editor employers: \$0--$\infty$.
\footnote{This would be an important number to estimate.  One possible approach is to write an algorithm which can match paid promotional edits with reasonable precision, then multiply the number of articles being edited by the average price per paid article, currently something like \$400.}
\item Declared paid editor employers: Unknown, see above.
\item Individual donors giving \$100 or less: \$65M or about 75\% of the WMF budget.
\footnote{\url{https://www.mediawiki.org/wiki/User:Adamw/Contribution_distribution}}
\footnote{\label{wmfaudit}\url{https://upload.wikimedia.org/wikipedia/foundation/d/da/Wikimedia_Foundation_Audit_Report_-_FY16-17.pdf}}
\item Unrestricted large grants of \$100 or more: \$22M or about 25\% of the WMF budget.
\item Restricted, large grants: Assets of \$3.5M
\footnote{WMF Audit FY16-17, section (5)}
\end{itemize}

\subsubsection{Where's the money?}

The Wikimedia Foundation holds \$113M in net assets.\cref{wmfaudit}

Wikimedia chapters hold [??]

\subsubsection{Who controls the money?}

The Wikimedia Foundation's Board has the final say over how their donor money will be spent.  Out of 10 members, 3 are nominally elected by a vote where about 4\% of active editors normally turn out.  2 members are chosen by chapters, and the remaining members are appointed by the Board itself.

\subsubsection{How does the money move?}



\subsubsection{Where does the money go?}

The Wikimedia Foundation's major spending is \$33.7M in payroll, and \$11.2M in grants.  \$24.2M more is spent on a mixed bag of services and other supporting costs.

Paid editor funds are dissipated by household expenses.

\subsubsection{"Fight for zero" against capitalist money}

There are efforts to get paid advertising out of Wikipedia, and undeclared paid advertising is already prohibited.

[give examples and statements]

\subsubsection{Fighting capitalist influence on the Wikimedia Foundation}

The low proportion of restricted grants and complete lack of advertising money is a great start for protecting the Wikimedia Foundation's independence.

\subsection{Knowledge}

Knowledge is the bread and butter of Wikipedia and its sister wiki projects[cn]. This is our currency and our only worth.

\subsubsection{What is knowledge?}

Good question.  For our purposes, it has two forms, either written down or retained in a living human.  Knowledge can be true or false.  It can be useful or irrelevant to the learner, depending on context.  Knowledges can be contradictory and yet coexist.  Knowledge has value.  Its use value can be decoupled from its exchange value.

The discussions on "talk" pages are the primary means of coordinating and communicating content creation, but we'll ignore as it's an intermediate product of knowledge production.

The corpus of all wikis consists of 351M articles.  13\% of these are in English, and the top 6 wikis hold half of all articles.

\subsubsection{Where does knowledge come from?}

Wikipedia [cn] claims that its knowledge should only come from secondary sources.  Ford [??] points out that the rules actually call for tertiary sources, yet actual sources are usually secondary (truly) or often (contraband) primary [???].

Some of the knowledge is original research, [meaning].

\subsubsection{Where is knowledge housed?}

Our corpus of knowledge is the written form stored on publicly available servers.  The Wikimedia Foundation hosts the dominant, editable interface to our corpus, but the raw data and forks are available, in the ballpark of [hundreds] on the Internet, and [thousands] of offline editions.

There is tremendous knowledge in the intersubjective minds of our current and future editors, and this guides their editing.  According to wiki rules, editor's personal experience isn't supposed to be directly transcribed into the corpus, it can only mediate work done with other sources of knowledge.

\subsubsection{Who produces knowledge?}

Editors and other wiki contributors are the only producers of knowledge.

\subsubsection{How is knowledge transferred?}

Content is added and edited mostly using the MediaWiki Web interface.  Roughly 5M new articles and 5M media files are added each year.

\subsubsection{Where does knowledge go?}

Readers find and consume our written corpus.  Doing so changes their internal state, hopefully in wonderful ways.  Articles must be discovered in order to be read.  [orphans].

Countless AIs and other downstream software devour the corpus in real-time, as it evolves.  There is no control or accounting for where this is going.  Our licenses don't require anything from these consumers.

\subsubsection{Fighting capitalist influence on knowledge}

[see fight for zero, above]

\subsubsection{Fighting freeloading capitalist use of knowledge}

The CC-SA and CC0 licenses allow virtually unlimited capitalist reuse of our knowledge.

We may have the alternative to use the non-commercial CC BY-NC-SA license, although [I don't know] this might not protect from indirect uses like training AIs owned by capitalists [cite showing that this might help, but also repeat lessons learned about for-profit unis, etc.]

With no requirement to give back or be accountable to the public, we don't know what the social impact of capitalist reuse might be.

\subsubsection{Expand the types of knowledge}

If we believe in the wiki model, then more types of knowledge should benefit from it.  For example, oral histories could be captured digitally and made available as primary sources.[Gallert]

\subsection{Labor}



\subsubsection{Who are the workers?}

[survey of unpaid vs paid]


\subsubsection{Who's the boss?}

Unpaid workers are independent.

Declared and shadow paid editors are on either short- or long-term contracts with their clients, who might be individuals or organizations.  

\subsubsection{Other power relationships between contributors}

[admin]

\subsubsection{Collaborative relationships between contributors}

[talk pages]

\subsubsection{Alienation}

Alienation is the lack of community between editors.  

On a larger scale, isolation between wiki language communities is another form of isolation.[self-focus]

\subsubsection{What are the tools of production?}

Editors own their own computers.  The Wikimedia Foundation owns servers and storage attached to their flagship domain names.  The Internet does the distribution.  Readers will need a networked device to consume.

Some editors rely on bot assistance

\subsubsection{Where is labor performed?}



\subsubsection{Who owns the tools of production?}



\subsubsection{What are the preconditions for knowledge production?}

At the moment, editors must have unhindered Internet access.  They must be literate, and if unpaid must have the free time to edit.

\subsubsection{What is being produced?}

[don't repeat knowledge section]
[how much is produced?]

\subsubsection{Fighting for workers' rights: an Editor's Union}

[history of]

\subsubsection{Fight against censorship and for net neutrality}

State and capitalist censorship or shaping of information is already a problem.

Slight changes in Google's search engine results pages can cause our site traffic to drop dramatically.

\subsubsection{Mitigate scarcity among preconditions}

[work backlogs and their side-effects]

\subsubsection{Counteract biases in contributor selection}

\section{Political economy of artificial intelligence}

AIs run by the Wikimedia Foundation (ORES) and by volunteers ([??]) are human-enhancing technologies, which exist to help contributors more effectively curate and produce.  The economically significant AIs are those that 

\section{Auditing}

[of AIs and the corpus]

\end{document}  